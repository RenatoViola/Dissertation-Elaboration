\typeout{NT FILE chapter6.tex}%

\chapter{Conclusion}
\label{cha:conclusion}

\epigraph{
  This chapter concludes the dissertation with a summary of the work developed throughout its course, revisiting the motivation, objectives and a synopsis of the proposed solution, outlining the features inspired by related work. Additionally, the system's shortcomings as well as its strengths are acknowledged via a discussion of the results of its two constituents. The chapter closes by envisioning the future of the developed system, contemplating further improvements and essential features.
}{}

Despite ongoing efforts by museums in promoting inclusivity, many showcases are yet predominantly visual, at the compromise of BVI visitors' independence and access to art. While competent, current accessibility methods are employed in exhibitions to differing degrees of success and often struggle to balance informational access and artistic engagement. In light of these limitations, the main objective of this research is to narrow the accessibility gaps persisted by the methods currently in place at museums, namely in what concerns the literal interpretation and aesthetic appreciation of paintings.

Informed by related work in the field of immersive audio systems applied in cultural settings and accessible virtual navigation tools, the solution proposed in this dissertation strives to meet the aforementioned purpose by way of a bipartite system. Its core constituent is a mobile soundscape player in which BVI users may autonomously navigate auditory painting recreations through several SATs, customizable sound effects (ambience and localized) and verbal descriptions. By combining sound stimuli and narrations, this hybrid approach merges both informational and aesthetic access to an artwork, while remaining highly interactive and immersive. Complementing this mobile solution by supplying it with content, is a desktop editor targeted at curators without prior sound design or programming experience. Via a simple grid-based system and interactive soundscape preview emulating the mobile application, it enables intuitive development of the auditory environments available in the soundscape player.

The preliminary interview with two blind individuals was essential in shaping what the mobile application's final iteration came to be, ensuring it addressed real BVI practical needs and preferences. Their feedback assisted in not only refining the directional scanner and other existing features, but also devise new ones such as the proximity sensor, ambience regulation and haptic feedback. The impact of this concept refinement phase was felt in the final user study's promising results, as BVI participants responded very positively to the features and refinements derived from those early interviews.

In the mobile soundscape player, tested solely by BVI individuals, most participants reported feeling immersed within the artwork and that they had a solid grasp on its content and essence. Valuing the existing audio cues, SATs and sound regulation options, they also felt capable of independent interaction with the application. Though some users commented on initial difficulties situating themselves within the scene, these were to be expected, lessening with practice and as SATs were subsequently introduced. Some users also called for more intense haptic feedback.

Similarly, the desktop editor was well regarded in its own separate study, tested by non-experts of differing backgrounds in place of curators or art professionals, due to time constraints and limited connections. Nonetheless, the participants successfully recreated a painting as an auditory environment, strongly considering the editor to be expressive and intuitive to work with. Though the implemented grid-based placement system was positively received and regarded as adequate, some users preferred alternatives such as drag-and-drop or point-and-click, while others mentioned augmentations to the current grid system. Furthermore, participants found the editor practical and functional yet somewhat unembellished in what pertained to common QoL features.

Despite it's general success, the proposed system has one major limitation, though fortunately it is neither conceptual nor did it interfere with any of the user tests. This deficiency manifests in the lack of an export capability from the editor to the mobile application, which was left unimplemented due to time constraints. While this feature is not critical to the dissertation's proof of concept, it does prevent the desktop application from effectively supporting the mobile player as the provider of its content. As such, it should be the first point to be addressed in a future iteration of the editor.

The proposed solution's shortcomings are greatly superseded by its merits, as evidenced by its very favourable reception from both target audiences. More than a simple amalgamation of proven SATs and audio techniques, the mobile soundscape player provided BVI users with an engaging and meaningful way to access visual art. Analogously, the editor empowered non-expert users to easily create immersive soundscapes of equivalent quality to the one present in the mobile prototype. While in the dissertation's current state barriers to accessibility still prevail in various forms, the work developed is a worthwhile step towards dismantling or at least attenuating them, presenting a strong foundation for what could eventually evolve into a widespread real-world solution.


\section{Future Work}
\label{sec:future-work}

\lipsum[1-5]