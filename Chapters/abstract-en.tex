%!TEX root = ../template.tex
%%%%%%%%%%%%%%%%%%%%%%%%%%%%%%%%%%%%%%%%%%%%%%%%%%%%%%%%%%%%%%%%%%%%
%% abstract-en.tex
%% NOVA thesis document file
%%
%% Abstract in English([^%]*)
%%%%%%%%%%%%%%%%%%%%%%%%%%%%%%%%%%%%%%%%%%%%%%%%%%%%%%%%%%%%%%%%%%%%

\typeout{NT FILE abstract-en.tex}%

In their mission as inclusive spaces of cultural participation and celebration, museums have taken considerable strides in overcoming their historical reliance on sight to adequately accommodate the needs of the \gls{BVI}. 

However, accessibility tends to be an afterthought rather than a priority, and most exhibitions have mainly remained inaccessible to a \gls{BVI} audience. Meritable as they are, the most common accessibility methods often fail to balance exposure to information and proper artwork engagement, if even available. Finding their independence, mobility, and interpretative access conditioned, in Europe, blind and low-vision people rarely attend these institutions despite enjoying and expressing the desire to experience visual art.

In this dissertation, we propose a remote and interactive approach to BVI-accessible visual art representation, in which spatial audio provides a feeling of immersion and simulates exploration. The devised system is symbiotically divided, with each part catering to a unique audience.

The system includes a mobile soundscape player, where \gls{BVI} users autonomously explore auditory environments representative of paintings through assisted thumbstick movement, directional audio cues, and a vicinity scanning \gls{SAT}, among other \gls{SAT}s. The experience offers room for customization, incorporating controls for sensory load regulation.
In addition, the mobile application is supplemented by a soundscape editor for Windows, in which museum curators without prior sound design expertise may quickly develop the immersive auditory scenes made available in the soundscape player to \gls{BVI} visitors.

Ultimately, this proposal aims to fulfill \gls{BVI} patrons' informational and aesthetic needs regarding visual art access, promoting their independence and tearing down barriers to accessibility.

% Palavras-chave do resumo em Inglês
% \begin{keywords}
% Keyword 1, Keyword 2, Keyword 3, Keyword 4, Keyword 5, Keyword 6, Keyword 7, Keyword 8, Keyword 9
% \end{keywords}
\keywords{
  Spatial Audio \and
  Blind and Visually Impaired \and
  Accessible Culture \and
  HRTF \and
  Soundscape
}
