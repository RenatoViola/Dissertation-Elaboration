%!TEX root = ../template.tex
%%%%%%%%%%%%%%%%%%%%%%%%%%%%%%%%%%%%%%%%%%%%%%%%%%%%%%%%%%%%%%%%%%%%
%% abstract-en.tex
%% NOVA thesis document file
%%
%% Abstract in English([^%]*)
%%%%%%%%%%%%%%%%%%%%%%%%%%%%%%%%%%%%%%%%%%%%%%%%%%%%%%%%%%%%%%%%%%%%

\typeout{NT FILE abstract-en.tex}%

In their mission as inclusive spaces of cultural participation and celebration, museums have taken considerable strides in overcoming their historical reliance on sight to adequately accommodate the needs of the \gls{BVI}. 

However, accessibility tends to be an afterthought rather than a priority, and most exhibitions have mainly remained inaccessible to a \gls{BVI} audience. Meritable as they are, the most common accessibility methods often fail to balance exposure to information and proper artwork engagement, if even available. Finding their independence, mobility, and interpretative access conditioned, in Europe, blind and low-vision people rarely attend these institutions despite enjoying and expressing the desire to experience visual art.

In this dissertation, a hybrid approach to BVI-accessible visual art interpretation is proposed to narrow current accessibility gaps by providing interaction with a painting's virtual sonic recreation through both sensorial and descriptive access to its contents. It is symbiotically divided into two components catering to unique audiences: BVI individuals and curators or art professionals.

The devised system comprises a mobile soundscape player, where \gls{BVI} users autonomously explore auditory environments representative of paintings through assisted thumbstick movement, directional audio cues, and a vicinity scanning \gls{SAT}, among other \gls{SAT}s. The application is supplemented by a desktop soundscape editor, with which curators without prior sound design expertise may quickly develop the immersive auditory scenes made available in the soundscape player for \gls{BVI} visitors.

Ultimately, this proposal aims to fulfill \gls{BVI} patrons' informational and aesthetic needs regarding visual art access, promoting their independence and tearing down barriers to accessibility. The system was evaluated via a distinct user study for each of its constituents, in which participants assessed the application's features and overall viability.

% Palavras-chave do resumo em Inglês
% \begin{keywords}
% Keyword 1, Keyword 2, Keyword 3, Keyword 4, Keyword 5, Keyword 6, Keyword 7, Keyword 8, Keyword 9
% \end{keywords}
\keywords{
  Blind and Visually Impaired \and
  Accessible Culture \and
  Spatial Audio \and
  Soundscape
}
