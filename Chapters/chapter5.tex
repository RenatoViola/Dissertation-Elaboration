%!TEX root = ../template.tex
%%%%%%%%%%%%%%%%%%%%%%%%%%%%%%%%%%%%%%%%%%%%%%%%%%%%%%%%%%%%%%%%%%%%
%% chapter5.tex
%% NOVA thesis document file
%%
%% Chapter with lots of dummy text
%%%%%%%%%%%%%%%%%%%%%%%%%%%%%%%%%%%%%%%%%%%%%%%%%%%%%%%%%%%%%%%%%%%%

\typeout{NT FILE chapter5.tex}%

\chapter{Evaluation and Results}
\label{cha:evaluation_and_results}

\epigraph{
  This chapter provides a detailed overview of the proposed system and briefly exposes the solution's validation methodology. It also addresses the expected technological stack and the envisioned plan for the system's development. The work schedule is split into five distinct and concisely explained tasks mapped in a Gantt chart.
}{}

\section{User Studies}

To evaluate the proposed system in its entirety, two final user studies were conducted, each corresponding to one of the two developed prototypes: the mobile soundscape player designed for Blind and Visually Impaired (BVI) users, and the Windows-based soundscape editor intended for museum curators without prior experience in sound design. The objective of these studies was to assess not only the technical functionality of each application but also their usability, accessibility, and the overall adequacy of the implemented features for their respective target audiences.

Despite addressing distinct user groups, both studies shared a common methodological structure to ensure consistency and comparability across results. Each followed a predefined protocol specifying the study’s general objectives, the equipment to be used (all provided by the researcher), the usage scenario and functionality overview, and the rules of researcher intervention during the sessions. In both cases, the researcher’s assistance was deliberately minimal, limited to clarifying questions about controls or task instructions, thereby allowing participants to interact with the applications as independently as possible.

The evaluation process combined structured and unstructured phases. Participants first completed a sequence of predefined tasks designed to cover the core functionalities of the respective application. Some tasks were highly specific (e.g., triggering a particular audio description or editing a sound source with defined parameters), while others allowed for more freedom in execution, encouraging natural exploration. This structured segment was followed by a free-use session, during which participants could interact with the system without predefined constraints. These sessions aimed to capture spontaneous behaviors and uncover usability issues not foreseen by the researcher.

At the conclusion of each session, participants filled out a survey consisting mainly of Likert-scale questions supplemented by open-ended qualitative prompts. The closed questions were designed to gauge perceived ease of use, clarity of functionality, and the perceived adequacy of the interaction model (e.g., joystick navigation, grid-based editing). The open-ended questions allowed participants to share broader impressions, report difficulties, and suggest improvements. Additionally, any spontaneous comments or notable observations made during the testing sessions were carefully recorded by the researcher and considered as part of the feedback.

The focus of the two protocols diverged according to the nature of the application. For the mobile soundscape player, emphasis was placed on accessibility, immersion, and the ability to meaningfully interpret a painting through sound. For the desktop soundscape editor, the main focus was on comprehensibility and intuitiveness, specifically whether non-expert users could successfully construct a functional and expressive auditory representation of a painting without prior technical training.

To guarantee anonymity and compliance with data protection requirements, each participant was assigned a numerical identifier, which is how they will be referenced in the subsequent analysis. Participation in both studies was voluntary and uncompensated.

\section{Mobile Sonic Painting Exploration}

\lipsum[1-5]

\subsection{User Characterization}

\lipsum[1-5]

\subsection{Task Overview}

\lipsum[1-10]

\subsection{Results Overview}

\lipsum[1-10]

\subsection{The Perspective of a "Returning" User}

\lipsum[1]

\section{Desktop Soundscape Creation}

\lipsum[1-5]

\subsection{User Characterization}

\lipsum[1-5]

\subsection{Task Overview}

\lipsum[1-10]

\subsection{Results Overview}

\lipsum[1-10]

\section{Summary}

\lipsum[1-4]
