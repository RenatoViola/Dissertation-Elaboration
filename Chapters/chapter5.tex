%!TEX root = ../template.tex
%%%%%%%%%%%%%%%%%%%%%%%%%%%%%%%%%%%%%%%%%%%%%%%%%%%%%%%%%%%%%%%%%%%%
%% chapter5.tex
%% NOVA thesis document file
%%
%% Chapter with lots of dummy text
%%%%%%%%%%%%%%%%%%%%%%%%%%%%%%%%%%%%%%%%%%%%%%%%%%%%%%%%%%%%%%%%%%%%

\typeout{NT FILE chapter5.tex}%

\chapter{Evaluation and Results}
\label{cha:evaluation_and_results}

\epigraph{
  This chapter provides a detailed overview of the proposed system and briefly exposes the solution's validation methodology. It also addresses the expected technological stack and the envisioned plan for the system's development. The work schedule is split into five distinct and concisely explained tasks mapped in a Gantt chart.
}{}

\section{User Studies}

To evaluate the proposed system in its entirety, two final user studies were conducted, each corresponding to one of the two developed prototypes: the mobile soundscape player designed for Blind and Visually Impaired (BVI) users, and the Windows-based soundscape editor intended for museum curators without prior experience in sound design. The objective of these studies was to assess not only the technical functionality of each application but also their usability, accessibility, and the overall adequacy of the implemented features for their respective target audiences.

Despite addressing distinct user groups, both studies shared a common methodological structure to ensure consistency and comparability across results. Each followed a predefined protocol specifying the study’s general objectives, the equipment to be used (all provided by the researcher), the usage scenario and functionality overview, and the rules of researcher intervention during the sessions. In both cases, the researcher’s assistance was deliberately minimal, limited to clarifying questions about controls or task instructions, thereby allowing participants to interact with the applications as independently as possible.

Participants first completed a sequence of predefined tasks designed to cover the core functionalities of the respective application. Some tasks were highly specific, while others allowed for more freedom in execution, encouraging natural interaction. The focus of the tasks wasn't evaluation in and of itself, but rather a pratical introduction to the system's features and controls. This structured segment was followed by free-use optional sessions, during which participants could continue to interact with the system without constraints. These sessions aimed to capture spontaneous behaviors and uncover usability issues not foreseen by the researcher, while familiarizing the user further with the system.

At the conclusion of each session, participants filled out a survey consisting mainly of Likert-scale questions supplemented by open-ended qualitative prompts. The closed questions were designed to gauge perceived ease of use, clarity of functionality, and the perceived adequacy of the interaction model. The open-ended questions allowed participants to share broader impressions, report difficulties, and suggest improvements. Additionally, any spontaneous comments or notable observations made during the testing sessions were carefully recorded by the researcher and considered as part of the feedback.

The focus of the two protocols diverged according to the nature of the application. For the mobile soundscape player, emphasis was placed on accessibility, immersion, and the ability to meaningfully interpret a painting through sound. For the desktop soundscape editor, the main focus was on comprehensibility and intuitiveness, specifically whether non-expert users could successfully construct a functional and expressive auditory representation of a painting without prior technical training. The approach defined in each protocol, allowed us to observe both task performance and also participants’ subjective experiences across the system's features.

To guarantee anonymity and compliance with data protection requirements, each participant was assigned a numerical identifier, which is how they will be referenced in the subsequent analysis. Participation in both studies was voluntary and uncompensated.


\section{Mobile Sonic Painting Exploration}

This section presents the evaluation of the mobile soundscape player, which enables BVI users to autonomously explore a painting through spatial audio, narration, haptic feedback, and orientation aids. It builds upon the general methodology outlined in Section 5.1, reinforcing the overall objective of assessing accessibility, immersion, and interpretability, while attending to the specific nuances of mobile interaction.

All user testing sessions were conducted on-site at the Raquel e Martin Sain Foundation — a social solidarity institution dedicated to the professional education and social integration of BVI individuals.
The Foundation provided the participants for this study and also the same fairly quiet room for all sessions. These conditions, alongside the consistent equipment setup and research protocol ensured a controlled environment with comparable conditions across participants.

The setup comprised a smartphone running the application, over-ear headphones, and a PlayStation 4 controller. This configuration was deliberately chosen to allow for two distinct modes of interaction: a physical, tangible control scheme through the gamepad and touch-based controls on the device itself. The headphones were needed for the reliable perception of spatial sound. After a quick overview of the application, participants were first introduced to the system through a series of structured tasks using the PS4 controller. As they were not yet acclimated to the application or its controls, this choice was intended to give them an initial sense of stability and control, as the tactile nature of the physical joystick and buttons provided a clear and accessible interaction model.

The evaluation scenario consisted of navigating a sonic rendition of "O Bosque Sagrado", a painting by Maria Benamor, guiding participants to locate specific sound sources, adjust playback settings, and make use of varying orientation aids. Following this guided exposure through the prototype's main features, participants were encouraged to explore the virtual soundscape more freely, initially continuing with the controller, then transitioning to touch controls, allowing for a direct comparison of both input modalities. 

The soundscape content was developed by the research team (in this case, the author), rather than by art professionals, using sound clips available online and Google Cloud's text-to-speech. Participants were informed of this ahead of the tasks, to manage expectations regarding narrative depth and artistic interpretation.

\subsection{User Characterization}

A total of eight participants took part in the evaluation of the mobile soundscape player. None of them had been previously exposed to the application or its functionalities, ensuring that their feedback reflected first-time user impressions rather than prior familiarity.

Participants’ ages ranged from 28 to 56 years old, with an average of 42.5 years. The age distribution is illustrated in Figure~\ref{fig:validation-age-distribution}, which shows that the sample was relatively balanced across middle adulthood, though no participants under 25 or above 60 were included.
\begin{figure}[htbp]
  \centering
  \includegraphics[width=0.8\linewidth]{goku}
  \caption{Age distribution.}
  \label{fig:validation-age-distribution}
\end{figure}

The group consisted of five male participants (62.5\%) and three female participants (37.5\%). In terms of visual condition, six participants (75\%) were blind and two participants (25\%) were classified as having low vision. Among the blind participants, some reported having lost their sight later in life rather than being blind from birth, which in some cases influenced the way they described their perception and interaction with auditory cues. Overall, the participant pool was diverse in terms of gender, age, and visual condition, and while small in absolute number, it provided a representative spectrum of BVI experiences relevant to evaluating the application’s accessibility.

When asked about their prior familiarity with gamepad controllers, most participants indicated that they were not fully accustomed to such devices. However, several noted that they had used them occasionally, particularly in the context of other user studies rather than for recreational gaming. This background influenced their degree of comfort with the PlayStation 4 controller used in the evaluation, though all participants were able to complete the required tasks with minimal guidance.

\subsection{Task Overview}

Participants were guided through a set of six tasks designed to gradually introduce the main features of the mobile soundscape player. As previously discussed in Section 5.1, these tasks were not conceived as a strict evaluation in themselves, but rather as a structured way of teaching participants how to use the system. They ensured that each participant had the opportunity to experience the core functionalities before moving on to the free exploration sessions.

The set of tasks was intentionally varied: while some were highly specific in scope (such as finding a particular sound source), others left room for interpretation and exploration, allowing participants to become more comfortable with the interface at their own pace. During subsequent tasks, participants were encouraged to use previously learned functionalities, but were free to ignore them as well.

Standing out from all six other tasks, Task 0 was exclusively intended for low-vision subjects, as it required a certain degree of residual vision to adjust the zoom level and toggle between different visual display modes. It was not presented to blind participants and actually carried little relevance even for those with low vision. Its optional character meant that it did not interfere with the main evaluation flow.

Task 1 focused on basic movement and spatial perception, introducing participants to the joystick-based control of direction and rotation, while Task 2 expanded on this by training orientation with respect to the original alignment of the painting, using the clock-based narration system. Task 3 introduced the directional scanner and the global narration feature, allowing users to hear localized descriptions of individual elements or a full scene overview. Task 4 presented the proximity sensor, which mimicked a car parking sensor by giving intermittent auditory feedback as the user approached a sound source. Task 5 allowed participants to regulate the number of active sound sources and toggle ambient sound, introducing the idea of controlling auditory complexity. Finally, Task 6 exposed participants to the two available playback modes — simultaneous and sequential — highlighting how these modes altered the listening experience.

Collectively, these tasks served as a step-by-step walkthrough of the application’s feature set. By progressively layering navigation, orientation, object identification, and sound management tools, participants were given a structured introduction that prepared them for the freer, less guided exploration sessions that followed.

\subsection{Results Overview}

\lipsum[1-10]

\subsection{The Perspective of a "Returning" User}

\lipsum[1]

\section{Desktop Soundscape Creation}

\lipsum[1-5]

\subsection{User Characterization}

\lipsum[1-5]

\subsection{Task Overview}

\lipsum[1-10]

\subsection{Results Overview}

\lipsum[1-10]

\section{Summary}

\lipsum[1-4]
