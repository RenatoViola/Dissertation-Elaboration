%!TEX root = ../template.tex
%%%%%%%%%%%%%%%%%%%%%%%%%%%%%%%%%%%%%%%%%%%%%%%%%%%%%%%%%%%%%%%%%%%%
%% chapter3.tex
%% NOVA thesis document file
%%
%% Chapter with a short latex tutorial and examples
%%%%%%%%%%%%%%%%%%%%%%%%%%%%%%%%%%%%%%%%%%%%%%%%%%%%%%%%%%%%%%%%%%%%

\typeout{NT FILE chapter3.tex}%

\makeatletter
\newcommand{\ntifpkgloaded}{%
  \@ifpackageloaded%
}
\makeatother


\chapter{Related Work}
\label{cha:related-work}

% epigraph configuration
\epigraphfontsize{\small\itshape}
\setlength\epigraphwidth{12.5cm}
\setlength\epigraphrule{0pt}

\epigraph{
  This chapter reviews relevant research and applications related to the topic, including technologies and approaches that have been used to address accessibility for visually impaired individuals. It identifies existing gaps and highlights the unique aspects of the proposed solution in comparison to current state-of-the-art techniques, as well as some similarities.
}{}

\section{Immersive Audio Experiences in Cultural and Learning Environments}

Objective: Review how immersive audio technologies have been applied to enhance engagement in cultural and educational settings.

Topics to Include:
Use of spatial audio, binaural audio, and Ambisonics in museums, galleries, and exhibitions.
Examples of audio-enhanced educational content, such as guided tours or virtual museum experiences.
Benefits of immersive audio for storytelling and conveying emotional or spatial dimensions of exhibits.
Case studies demonstrating the impact of these technologies on user engagement.

Link to Dissertation: Highlight the potential of immersive audio as a tool for BVI accessibility and artistic appreciation.

\subsection{Navmol}

Navmol~\cite{fartaria2013navmol} is a molecular browser and editor specifically designed for blind and visually impaired users, aiming to provide BVI accessibility in the higher education of chemistry. Via a speech synthesizer, it provides an auditory portrayal of the atomic composition of complex molecular structures. Such configurations are bidimensionally depicted using the analogue clock metaphor, consisting of mapping directions to the positions of a clock and well known among the BVI community. As some users retain some degree of sight, the program has a simple graphical interface (Figure~\ref{fig:navmol-img}) with them in mind, visually displaying the selected atoms and some molecular contours.
\begin{figure}[htbp]
  \centering
  \includegraphics[width=0.6\linewidth]{navmol}
  \caption{Navmol's clock system, where carbon-3 has neighbors at 6 o'clock and 10 o'clock, respectively carbon-4 and carbon-2 (Adapted from~\cite{fartaria2013navmol}).}
  \label{fig:navmol-img}
\end{figure}

Rodrigues~\cite{rodrigues2015navmol} merged the simple clock analogy with the application of HRTFs on the auditory signal generated by the existing Navmol program at the time (version 2.0), in order to create realistic directional sound cues, perceived to derive from where the atom is positioned. Usability tests demonstrated the efficacy of this integration, with users achieving an average task accuracy of 95.7\% in identifying and navigating molecular structures.

Knowing that HRTF performance may greatly vary across users due to inter-individual differences in morphology, Rodrigues~\cite{rodrigues2015navmol} conducted a study on the performance of 53 distinct HRTF measurements. It confirmed a significant variation in performance across different HRTF datasets for individual users. The five most consistently well performing measurements – KEMAR, CIAIR, IRC05, IRC25 and IRC44, were then selected for use in one additional study, motivated by the significant variation in performance across different HRTF datasets for individual users. The necessity to allow users to select their preferred HRTF dataset was made apparent and Navmol was updated accordingly.

Rodrigues’~\cite{rodrigues2015navmol} work very much aligns with the goals of this dissertation, as it highlights not only the efficacy of spatial audio in the perception of spatial and structural information but also addresses the suitability (or lack thereof) of specific HRTFs to distinct users, allowing them to tailor their own experience to a degree.


\section{Accessibility in Auditory Representations of Art and Virtual Environments}

Objective: Discuss efforts to make art and virtual environments accessible through auditory means.

Topics to Include:
Audio descriptions of visual art and their focus on interpretation and historical context.
Research on spatialized audio as a method to convey spatial and emotional dimensions of visual art.
Applications in virtual environments that support BVI navigation and exploration through sound.
Studies comparing the effectiveness of auditory-only interfaces to multisensory approaches.

Link to Dissertation: Identify gaps in auditory representation methods for BVI individuals, emphasizing the need for interactive, spatially dynamic solutions.

\section{Leisurely and BVI Inclusive Applications}

Objective: Explore how gamification and accessible design principles have been used to engage BVI individuals in leisure and entertainment.

Topics to Include:
BVI-accessible video games incorporating spatial audio for navigation and storytelling.
Gamified applications that support learning or cultural engagement for BVI users.
Examples of audio-only games and their design considerations for accessibility.

Link to Dissertation: Highlight how gamification principles and interactive audio can enhance both accessibility and user engagement in art-related experiences.

\section{Tools for Soundscape Creation}

Objective: Review existing tools and frameworks for creating soundscapes, with a focus on their accessibility and ease of use.

Topics to Include:
Software tools used for designing soundscapes (e.g., DAWs with spatial audio plugins, dedicated soundscape editors).
Approaches for integrating spatial audio and interactivity into soundscape creation.
Evaluation of user-friendliness and adaptability of these tools for non-experts, such as museum curators.

Link to Dissertation: Emphasize the need for intuitive, accessible tools tailored to curators and the creation of high-quality, interactive soundscapes.

\section{Summary and Final Design Considerations}

Objective: Provide a synthesis of the reviewed work, highlighting gaps and considerations relevant to your proposed solution.

Topics to Include:
Key limitations of existing immersive audio systems and accessibility efforts.
Opportunities for innovation in soundscape design and spatial audio accessibility.
Final considerations for designing a tool that balances accessibility, interactivity, and usability for both creators and BVI users.

Link to Dissertation: Summarize how your proposed solution builds upon and addresses the shortcomings identified in prior research.