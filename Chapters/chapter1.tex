%!TEX root = ../template.tex
%%%%%%%%%%%%%%%%%%%%%%%%%%%%%%%%%%%%%%%%%%%%%%%%%%%%%%%%%%%%%%%%%%%
%% chapter1.tex
%% NOVA thesis document file
%%
%% Chapter with introduction
%%%%%%%%%%%%%%%%%%%%%%%%%%%%%%%%%%%%%%%%%%%%%%%%%%%%%%%%%%%%%%%%%%%

\typeout{NT FILE chapter1.tex}%

\chapter{Introduction}
\label{cha:introduction}

\prependtographicspath{{Chapters/Figures/Covers/}}

% epigraph configuration
\epigraphfontsize{\small\itshape}
\setlength\epigraphwidth{12.5cm}
\setlength\epigraphrule{0pt}

\includegraphics[width=0.1\linewidth]{NOVAthesisFiles/Images/novathesis-insignia}\hfill
\includegraphics[width=0.875\linewidth]{NOVAthesisFiles/Images/novathesis-text}

% \noindent This is the \gls{novathesis} \LaTeX\ template \ntindex[Template!]{Version} \novathesisversion\ from   {Template!date}\novathesisdate.

\epigraph{
  This work is licensed under the \href{https://www.latex-project.org/lppl/lppl-1-3c/}{\LaTeX\ Project Public License v1.3c}.
  To view a copy of this \ntindex[Template!]{license}, visit the \href{https://www.latex-project.org/lppl/}{LaTeX project public license}.
}

In a colorful world brimming with life and filled with the most varied shapes and patterns, each of our five senses is vital in obtaining information about our surroundings~\cite{five-senses}. Out of these, vision is by far the most dominant, as about 80\% of what we learn about the world and the very impressions we perceive are by means of our sight~\cite{vision-is-dominant-sense}. It is so valued in fact, that not only 77\% of people state that it is their most important sense, but would also rather live shorter lives than a longer one without their sight~\cite{sight-is-most-important-sense-1, sight-is-most-important-sense-2}.

Unfortunately, sight is not a universal privilege. According to the International Agency for the Prevention of Blindness (IAPB) Vision Atlas~\cite{global-estimates-of-vision-loss}, as of 2020, approximately 1.1 billion people worldwide were living with some form of vision loss. While over the last few decades there has been a decrease of vision loss prevalence in proportion across the population, the absolute numbers have increased over time and don't seem to be dwindling anytime soon. On the contrary, all categories of vision loss are set to grow as it is projected that by 2050, the world will have roughly 1.8 billion people experiencing visual impairment~\cite{vision-loss-projections-2020-2050}. This is an increase in vision loss of 55\%, or 600 million people over the next 30 years.

If inclusivity and disability rights~\cite{crpd-rights} alone aren’t enough motivation to care for this community, the World’s Health Organization (WHO)~\cite{who-vision-impairment} states that everyone, if they live long enough, will at least experience some sort of eye condition in their lifetime in need of proper care. As such, the concerns of the visually impaired community should in fact be everyone's concerns as well.


\section{Motivation}
\label{sec:motivation}

Art is a universal form of human expression, be it cultural, creative, somewhere in between, or something else entirely. From their very conception, musems have traditionally been one of, if not the most prominent way to not only access such art, but to celebrate and share it across generations. These institutions are indeed known for being the custodians of history, heritage and artistic expression.

It is the very mission of a museum to be an inclusive and accessible space of cultural participation, a shared human experience. However, that is not always the case. Exhibitions have historically been visual in nature, as most of the art made available is specifically designed to be consumed that way and due to numerous efforts in preserving it. This is a huge barrier for BVI people in terms of access to information and even hinders their mobility and independence within an exhibition~\cite{cavazos2021accessible, li2023understanding, vasilakou2022accessibility, vaz2020blind, vaz2020perspectives}.

Though the visually impaired are still not accommodated properly to this day, there have been strides in the right direction, one of which is in the form of legislation advocating for their right to participate in cultural life~\cite{holloway2019making,li2023understanding,martins2020blindness} – article 30 of the UN Convention on the Rights of Persons with Disabilities~\cite{crpd-rights}. In recent times, museums too have made considerable efforts in adapting to the needs of the visually impaired aside from mobility and navigation, gradually becoming more inclusive and participatory spaces~\cite{rector2017eyes,li2023understanding,martins2020blindness,holloway2019making,asakawa2019independent}. These efforts tend to be expressed as: tactile replicas and graphics, pre-recorded audio descriptions, specialized tours, workshops, large print and labels in braille, among several others. Though each of these methods serves a relevant and unique purpose, they are not without limitations, which will be enumerated in section~\ref{sec:objectives}.

Not to undermine the progress achieved over the last few years, as institutions are now more accessible than they ever were, most museums are still largely unnaccessible to a BVI audience, as accessibility is more of an after thought than a priority~\cite{candlin2003blindness,holloway2019making,rector2017eyes,vasilakou2022accessibility}. As a result, low-vision people very rarely attend these institutions - in Europe, only 5.5\% of them actually do~\cite{vaz2020blind,vaz2020perspectives}. This is particularly due to the aforementioned ocular centricity which greatly limits not only their mobility but also their access to information, which in turn greatly hinders their independence as well~\cite{li2023understanding,vasilakou2022accessibility,cavazos2021accessible}.

Such a low percentual for museum adherence is particularly demoralizing, as it is a common observation that BVI people actually do enjoy and express the desire to visit galleries and experience visual art~\cite{candlin2003blindness,krol2024design,holloway2019making,li2023understanding,asakawa2019independent}, and much for the same reasons people with sight do~\cite{candlin2003blindness}. However, they must be provided with the proper acess to do so~\cite{holloway2019making} and don’t want to constantly rely on others~\cite{krol2024design,asakawa2019independent}.

While the numerous accessibility methods currently employed at musems merit their very welcome addition to any exhibition, these often tend to understate the importance of the aestheticity in experiencing the artwork~\cite{krol2024design,li2023understanding,martins2020blindness}. Instead, the focus leans heavily on describing and educating, sometimes at the cost of the sensory, emotional and immersive dimensions of a piece. Good accessibility requires a fine-tuned balance of both exposure to information and actual engagement with the artwork. This is especially important for visually impaired visitors, as it offers them the opportunity to develop meaninful connections to art, going beyond the pure intellectual understanding of it~\cite{martins2020blindness}. Approaches combining music and soundscapes have shown promise in this regard, but can be costly in terms of time and resources~\cite{krol2024design}.

Technological innovations such as spatial audio, haptics, VR and AR have enabled the creation of increasingly immersive experiences~\cite{chang2024sound,li2024beyond,yang2019audio,sanchez2007usability}, allowing not only for multisensory engagement with art but also opportunities for remote artistic appreciation. Despite the risk of less engagement than in person~\cite{li2023understanding} and difficulty navigating through a web gallery~\cite{virtual-space-accessibility}, digital platforms and virtual tours can tear down physical barriers by enabling individuals to experience art from the comfort of home or anywhere else, at their own pace, and unschackled by social tension~\cite{li2023understanding,how-museums-remove-barriers-for-bvi}. 

The potential of virtual environments to provide a variety of interaction modalities and content has been thoroughly investigated and sound-based approaches are especially important for blind and visually impaired people~\cite{yang2019audio,sanchez2007usability}. By mimicking environmental cues, spatial audio improves spatial orientation and makes it easier for users to move freely and independently in virtual environments, helping them visualize virtual worlds~\cite{sanchez2007usability}. Frequently enhanced with 3D effects to improve spatial perception and experience, 3D audio is particularly useful for navigation and immersive engagement in museum and gallery settings~\cite{yang2019audio}.

Mostly supporting the above mentioned technologies, the smartphone is the perfect vehicle for accessibility as about 54\% of the global population owns at least one~\cite{smartphone-stats} and it is rich in accessibility features among several others~\cite{smartphone-features}. This study draws inspiration from several implementations of accessible technologies to develop a remote mobile and surround sound-based solution, that addresses both the informational and aesthetic needs of BVI visitors.



\section{Problem description \& objectives}
\label{sec:objectives}

As it was briefly alluded to in section~\ref{sec:motivation}, there are some limitations to the accessibility methods usually active at museums~\cite{holloway2019making,rector2017eyes,candlin2003blindness,cavazos2021accessible}. Without delving into too much detail, tours and workshops are infrequent and inconsistent, must be reserved in advance and are only available on specific dates or timeslots. Though more common in museums, audio descriptions are primarily designed with normovisual people in mind, focusing mostly on intepretation and historical context, not accessibility. Braille-based brochures leave a great deal of relevant information aside, and braille proficiency is generally low~\cite{cavazos2021accessible}.

While a tactile approach seems to be the preferred form of interaction with artwork for BVI individuals, since it allows them to feel the artwork “up close and personal" sensing its various features at a low level, high level information about the piece is quite limited and the combination of preservation efforts and barriers of intellectual access with the still prevalent visual centricity of exhibitions makes it so that these types of programs are a rare occurrence among museums~\cite{li2023understanding,}.

In the context of this dissertation, we intend to tap into the potential of spatial audio as a means of creating an immersive remote experience that can convey the spatial and emotional dimensions of art, through interactive three-dimensional soundscapes. To achieve this, we draw inspiration from the use of spatial audio in other research as well as in video games and their accessibility features. Although video games may not be the first medium that comes to mind when considering the experience of visual art in museums, they share certain parallels. As an art form and medium of expression of its own, video games focus on immersion and engagement, much like museums aim to captivate their visitors. Over time, video games have also evolved to become more accessible and appeal to a broader audience through the innovative use of spatial audio to guide navigation, evoke emotion and tell a rich, interactive story. This aligns closely with our objectives and while our proposal is not to develop a video game, it adopts a gamified approach to reimagining how art can be experienced inclusively.

Thus, we propose the creation of a tool allowing museum curators to build a spatial audio environment representing art pieces, enabling the staff to define which parts of the environment are explorable by the visitors. In essence, it is an intuitive and user-friendly soundscape editor with support for immersive audio and interactivity, not requiring prior experience in more sophisticated tools with a harsh learning curve. The generated environments are then made available via a mobile application where blind and visually impaired clients can navigate a simplified top-down map-like view of the scene with contrasting elements, either using thumb joysticks or the phone’s location and gyroscope to define direction and moviment. With only a smartphone coupled with a pair of headphones, any user is able to explore immersive 3D sound simulating the experience of physically approaching or moving away from different parts of the artwork, using controls that are both basic and familiar.


\section{Expected contributions}


\section{Document Structure}

\section{Welcome to the \novathesis\ Template}
\label{sec:if_you_use_this_template}

This first Chapter introduces the \gls{novathesis} template and how it is organized. In Chapter~\ref{cha:users_manual} you can find some specific instructions on how to use this template.  Chapter~\ref{cha:a_short_latex_tutorial_with_examples} shows some examples and give some hints on how to write your text. Please read these next Chapters carefully.

\subsection{Your Time is Precious}
\label{sub:time_is_money}

Did you learn how to drive by sitting by the wheel and throwing your car into the road?  Most probably you did take your time \emph{learning the rules} and \emph{practicing} first! Likewise, it is not wise to throw yourself at the task of writing a thesis/dissertation in \LaTeX\ without seriously considering the following \ntindex{recommendation}!

\begin{tcolorbox}[colback=green!8]
  If you are going to spend zillions of hours writing your thesis/dissertation using the \gls{novathesis} \LaTeX\ template (or some other \LaTeX\ template), be wise and spend a couple of hours learning how to use it properly by reading its manual.  And then, be even wiser, and spend a few more hours \href{https://github.com/joaomlourenco/novathesis/wiki\#learning-latex}{learning some \LaTeX}.  I am sure that the time you are investing now will pay itself countless times before you submit your thesis/dissertation.\\\parbox{\linewidth}{\raggedleft---~\emph{João Lourenço}}
\end{tcolorbox}

\subsection{Recognition}
\label{sub:recognition}

\ntindex[Recognition]{}

The \gls{novathesis} template was born in~1996, and what you see now accumulates to many many hundreds (thousands?!) of working hours, unpaid and stolen from family and friends.  This work is available to the community under the \href{LaTeX project public license}{\LaTeX\ Project Public License v1.3c}, which means you are entitled to use it for free and change it at your will.  However, if you decide to use this template to write your thesis/dissertation, \textbf{be fair to the developers} and:
\begin{enumerate}
  \item \ntindex[novathesis!Citation]{} Cite the \gls{novathesis} manual~\cite{novathesis-manual} in a place of your choice (e.g., in the \emph{Acknowledgments}) of your thesis/dissertation with “\verb!\cite{novathesis-manual}!” .  If you cite it this way, the correct entry will be added automatically to your bibliography (no need to worry with the necessary BibTeX entry, as it will be added automatically);
  \item Go to the
\href{https://github.com/joaomlourenco/novathesis}{\ntindex[GitHub!project web page]{project web page} in GitHub} and give the project a \ntindex[GitHub!stars]{star} (marked with a red ellipse at the top-right in Figure~\ref{fig:github}); and
  \item Make a \ntindex[donations]{donation} by visiting the \gls{novathesis} project page and clicking in the button marked with a green ellipse at the top-center in Figure~\ref{fig:github}).  Alternatively, just click \href{https://www.paypal.com/donate/?hosted_button_id=8WA8FRVMB78W8}{\fcolorbox{DarkGreen}{gray!15}{\textbf{~HERE~}}} and your browser will be directed to the right page.
\end{enumerate}

\begin{figure}[htbp]
    \centering
    \includegraphics[width=0.5\linewidth]{github1}
    \caption{The \gls{novathesis} project web page in GitHub.}
    \label{fig:github}
\end{figure}

\section{The \emph{NOVAthesis} Template}
\label{sec:a_bit_of_history}

\ntindex[Template]{}

The \gls{novathesis} template was born at the \gls{DI} of  \gls{FCT} of \gls{NOVA}, Portugal.  But the user base grew… initially grew to other Departments of FCT-NOVA, then to other Schools of NOVA, and later to other Schools of other Universities.  Currently more than~25 Schools are natively supported by the \gls{novathesis} template (see Tables~\ref{tab:supported_schools_NOVA University Lisbon}, \ref{tab:supported_schools_University of Lisbon}, \ref{tab:supported_schools_University of Minho}, \ref{tab:supported_schools_Instituto Politécnico de Lisboa}, \ref{tab:supported_schools_Instituto Politécnico de Setúbal}, and \ref{tab:supported_schools_Other Universities/Schools/Degrees}).

\newenvironment{ntUniversity}[1]{
  \renewcommand\tabularxcolumn[1]{m{##1}}% for vertical centering text in X column
  % \renewcommand\cellgape{\Gape[1cm]}
  % \setcellgapes{20pt}
  % \makegapedcells
  % % \setlength{\extrarowheight}{20pt}
  % \renewcommand{\arraystretch}{2}
  \rowcolors{1}{}{GhostWhite}
    \xltabular{\linewidth}{cX}%
      \caption{#1's Schools supported by the \gls{novathesis} template\label{tab:supported_schools_#1}}\\
    \toprule%
    \rowcolor{Gainsboro}%
    & \Gape[1.5ex]{\thead[l]{#1}}\\
    \midrule%
}{%
    \bottomrule
    \endxltabular%
}

\makeatletter
\newtoggle{coverspace}
\newcommand{\docCover}[1]{%
  \setlength{\fboxsep}{0pt}%
  \togglefalse{coverspace}%
    \Gape[1.5ex]{\begin{mcellbox}[cc]
    \@for\myi:=#1\do{%
      \fbox{\colorbox{White}{\includegraphics[align=c,width=1.5cm]{1up/\myi}}}%
        \ifx\@xfor@nextelement\@nnil
          % last iteration
        \else
          % not last iteration
          \iftoggle{coverspace}{\togglefalse{coverspace}\\\\[-14pt]}{\toggletrue{coverspace}~}%
        \fi
  }%
    \end{mcellbox}}
}
\makeatother
\newcommand{\schlName}[3]{\textbf{#1} (\href{#3}{#2})}
\newcommand{\degreeName}[3]{\newline\null\quad • #1 \href{#3}{(#2)}}

\begin{ntUniversity}{NOVA University Lisbon}
  {
      \docCover{nova-fct-phd-en,nova-fct-msc-en}
  } &  {
    \schlName{NOVA School of Science and Technology}{FCT-NOVA}{https://www.fct.unl.pt}
    \degreeName{All PhD Programs}{PhD}{https://www.fct.unl.pt/en/education/phd-programmes}
    \degreeName{All MSc Programs}{MSc}{https://www.fct.unl.pt/en/education/master-degrees}
  }\\
  {
      \docCover{nova-fcsh-phd-en,nova-fcsh-msc-en}
  } &  {
    \schlName{NOVA School of Social Sciences and Humanities}{FCSH-NOVA}{https://www.fcsh.unl.pt}
    \degreeName{All PhD Programs}{PhD}{https://www.fcsh.unl.pt/cursos/\#doutoramentos}
    \degreeName{All MSc Programs}{MSc}{https://www.fcsh.unl.pt/cursos/\#mestrados}
  }\\
  {
      \docCover{nova-ims-phd-en,nova-ims-msc-mmaa-en,%
                nova-ims-msc-megi-en,nova-ims-msc-mgi-en,%
                nova-ims-msc-mcsig-en,nova-ims-msc-mgt-en}
  } &  {
    \schlName{NOVA Information Management School}{NOVA-IMS}{https://www.novaims.unl.pt}
    \degreeName{All PhD Programs}{PhD}{https://www.novaims.unl.pt}
    \degreeName{Master's in Data Science and Advanced Analytics}{MMAA}{https://www.novaims.unl.pt/mdsaa-ba}
    \degreeName{Master's in in Statistics and Information Management}{MEGI}{https://www.novaims.unl.pt}
    \degreeName{Master's in Information Management}{MGI}{https://www.novaims.unl.pt}
    \degreeName{Master's in Geographic Information Systems and Science}{MCSIG}{https://www.novaims.unl.pt/unigis}
    \degreeName{Master's in Geospatial Technologies}{GeoTech}{https://www.novaims.unl.pt/geotech}
  }\\
  {
      \docCover{nova-ensp-phd-en,nova-ensp-msc-en}
  } &  {
    \schlName{National School of Public Heath}{ENSP-NOVA}{https://www.ensp.unl.pt}
    \degreeName{All PhD Programs}{PhD}{https://www.ensp.unl.pt/courses/phds}
    \degreeName{All MSc Programs}{MSc}{https://www.ensp.unl.pt/courses/masters}
  }\\
  {
      \docCover{nova-itqb-phd-green-en,nova-itqb-msc-green-en,%
                nova-itqb-phd-gray-en,nova-itqb-msc-gray-en}
  } & {
    \schlName{Instituto de Tecnologia Química e Biológica}{ITQB-NOVA}{https://www.itqb.unl.pt}
    \degreeName{All PhD Programs}{PhD}{https://www.itqb.unl.pt/education/PhD\%20Programs}
    \degreeName{All MSc Programs}{MSc}{https://www.itqb.unl.pt/education/masters-courses}
  }\\
\end{ntUniversity}

\begin{ntUniversity}{University of Lisbon}
  {
    \docCover{ulisboa-ist-phd-en,ulisboa-ist-msc-en}
  } & {
    \schlName{Instituto Superior Técnico}{IST-UL}{https://tecnico.ulisboa.pt}
    \degreeName{All PhD Programs}{PhD}{https://tecnico.ulisboa.pt/en/education/courses/phd-programmes/}
    \degreeName{All MSc Programs}{MSc}{https://tecnico.ulisboa.pt/en/education/courses/masters-programmes}
  }\\
  {
    \docCover{ulisboa-fc-phd-en,ulisboa-fc-msc-en}
  } & {
    \schlName{Faculdade de Ciências}{FCUL}{https://ciencias.ulisboa.pt}
    \degreeName{All PhD Programs}{PhD}{https://ciencias.ulisboa.pt/en/educational-offer/}
    \degreeName{All MSc Programs}{MSc}{https://ciencias.ulisboa.pt/en/educational-offer/}
  }\\
  {
    \docCover{ulisboa-fmv-phd-en,ulisboa-fmv-msc-en}
  } & {
    \schlName{Faculdade de Medicina Veterinária}{FMV-UL}{https://www.fmv.ulisboa.pt}
    \degreeName{All PhD Programs}{PhD}{https://www.fmv.ulisboa.pt/en/study/phd}
    \degreeName{All MSc Programs}{MSc}{https://www.fmv.ulisboa.pt/pt/ensino/mestrados}
  }\\
\end{ntUniversity}


\newdata*{schlname}
\newdata*{schlurl}
\schlname(ea):={School of Architecture}
\schlurl(ea):={https://www.uminho.pt/EN/uminho/Units/schools-and-institutes/Pages/School-of-Architecture.aspx}
\schlname(ec):={School of Sciences}
\schlurl(ec):={https://www.uminho.pt/EN/uminho/Units/schools-and-institutes/Pages/school-of-sciences.aspx}
\schlname(ed):={School of Law}
\schlurl(ed):={https://www.uminho.pt/EN/uminho/Units/schools-and-institutes/Pages/school-of-law.aspx}
\schlname(ee):={School of Engineering}
\schlurl(ee):={https://www.uminho.pt/EN/uminho/Units/schools-and-institutes/Pages/school-of-Engineering.aspx}
\schlname(eeg):={School of Economics and Management}
\schlurl(eeg):={https://www.uminho.pt/EN/uminho/Units/schools-and-institutes/Pages/school-of-economics-and-management.aspx}
\schlname(em):={School of Medicine}
\schlurl(em):={https://www.uminho.pt/EN/uminho/Units/schools-and-institutes/Pages/school-of-Medicine.aspx}
\schlname(ep):={School of Psychology}
\schlurl(ep):={https://www.uminho.pt/EN/uminho/Units/schools-and-institutes/Pages/school-of-psychology.aspx}
\schlname(ese):={School of Nursing}
\schlurl(ese):={https://www.uminho.pt/EN/uminho/Units/schools-and-institutes/Pages/school-of-nursing.aspx}
\schlname(i3b):={Research Institute 13Bs}
\schlurl(i3b):={https://www.uminho.pt/EN/uminho/Units/schools-and-institutes/Pages/Institute-I3Bs.aspx}
\schlname(ics):={Institute of Social Sciences}
\schlurl(ics):={https://www.uminho.pt/EN/uminho/Units/schools-and-institutes/Pages/institute-of-social-sciences.aspx}
\schlname(ie):={Institute of Education}
\schlurl(ie):={https://www.uminho.pt/EN/uminho/Units/schools-and-institutes/Pages/institute-of-education.aspx}
\schlname(elach):={School of Arts and Humanities}
\schlurl(elach):={https://www.uminho.pt/EN/uminho/Units/schools-and-institutes/Pages/school-of-arts-and-humanities.aspx}


\begin{ntUniversity}{University of Minho}
    {
      \docCover{uminho-ea-phd-en,uminho-ea-msc-en}
    } & {
      \schlName{\theschlname(ea)}{\uppercase{ea}-UMINHO}{\theschlurl(ea)}
      \degreeName{All PhD Programs}{PhD}{https://www.uminho.pt/EN/education/educational-offer/Pages/PhD-degrees.aspx}
      \degreeName{All MSc Programs}{MSc}{https://www.uminho.pt/EN/education/educational-offer/Pages/Master-degrees.aspx}
    }\\
    {
      \docCover{uminho-ec-phd-en,uminho-ec-msc-en}
    } & {
      \schlName{\theschlname(ec)}{\uppercase{ec}-UMINHO}{\theschlurl(ec)}
      \degreeName{All PhD Programs}{PhD}{https://www.uminho.pt/EN/education/educational-offer/Pages/PhD-degrees.aspx}
      \degreeName{All MSc Programs}{MSc}{https://www.uminho.pt/EN/education/educational-offer/Pages/Master-degrees.aspx}
    }\\
    {
      \docCover{uminho-ed-phd-en,uminho-ed-msc-en}
    } & {
      \schlName{\theschlname(ed)}{\uppercase{ed}-UMINHO}{\theschlurl(ed)}
      \degreeName{All PhD Programs}{PhD}{https://www.uminho.pt/EN/education/educational-offer/Pages/PhD-degrees.aspx}
      \degreeName{All MSc Programs}{MSc}{https://www.uminho.pt/EN/education/educational-offer/Pages/Master-degrees.aspx}
    }\\
    {
      \docCover{uminho-ee-phd-en,uminho-ee-msc-en}
    } & {
      \schlName{\theschlname(ee)}{\uppercase{ee}-UMINHO}{\theschlurl(ee)}
      \degreeName{All PhD Programs}{PhD}{https://www.uminho.pt/EN/education/educational-offer/Pages/PhD-degrees.aspx}
      \degreeName{All MSc Programs}{MSc}{https://www.uminho.pt/EN/education/educational-offer/Pages/Master-degrees.aspx}
    }\\
    {
      \docCover{uminho-eeg-phd-en,uminho-eeg-msc-en}
    } & {
      \schlName{\theschlname(eeg)}{\uppercase{eeg}-UMINHO}{\theschlurl(eeg)}
      \degreeName{All PhD Programs}{PhD}{https://www.uminho.pt/EN/education/educational-offer/Pages/PhD-degrees.aspx}
      \degreeName{All MSc Programs}{MSc}{https://www.uminho.pt/EN/education/educational-offer/Pages/Master-degrees.aspx}
    }\\
    {
      \docCover{uminho-em-phd-en,uminho-em-msc-en}
    } & {
      \schlName{\theschlname(em)}{\uppercase{em}-UMINHO}{\theschlurl(em)}
      \degreeName{All PhD Programs}{PhD}{https://www.uminho.pt/EN/education/educational-offer/Pages/PhD-degrees.aspx}
      \degreeName{All MSc Programs}{MSc}{https://www.uminho.pt/EN/education/educational-offer/Pages/Master-degrees.aspx}
    }\\
    {
      \docCover{uminho-ep-phd-en,uminho-ep-msc-en}
    } & {
      \schlName{\theschlname(ep)}{\uppercase{ep}-UMINHO}{\theschlurl(ep)}
      \degreeName{All PhD Programs}{PhD}{https://www.uminho.pt/EN/education/educational-offer/Pages/PhD-degrees.aspx}
      \degreeName{All MSc Programs}{MSc}{https://www.uminho.pt/EN/education/educational-offer/Pages/Master-degrees.aspx}
    }\\
    {
      \docCover{uminho-ese-phd-en,uminho-ese-msc-en}
    } & {
      \schlName{\theschlname(ese)}{\uppercase{ese}-UMINHO}{\theschlurl(ese)}
      \degreeName{All PhD Programs}{PhD}{https://www.uminho.pt/EN/education/educational-offer/Pages/PhD-degrees.aspx}
      \degreeName{All MSc Programs}{MSc}{https://www.uminho.pt/EN/education/educational-offer/Pages/Master-degrees.aspx}
    }\\
    {
      \docCover{uminho-ics-phd-en,uminho-ics-msc-en}
    } & {
      \schlName{\theschlname(ics)}{\uppercase{ics}-UMINHO}{\theschlurl(ics)}
      \degreeName{All PhD Programs}{PhD}{https://www.uminho.pt/EN/education/educational-offer/Pages/PhD-degrees.aspx}
      \degreeName{All MSc Programs}{MSc}{https://www.uminho.pt/EN/education/educational-offer/Pages/Master-degrees.aspx}
    }\\
    {
      \docCover{uminho-ie-phd-en,uminho-ie-msc-en}
    } & {
      \schlName{\theschlname(ie)}{\uppercase{ie}-UMINHO}{\theschlurl(ie)}
      \degreeName{All PhD Programs}{PhD}{https://www.uminho.pt/EN/education/educational-offer/Pages/PhD-degrees.aspx}
      \degreeName{All MSc Programs}{MSc}{https://www.uminho.pt/EN/education/educational-offer/Pages/Master-degrees.aspx}
    }\\
    {
      \docCover{uminho-elach-phd-en,uminho-elach-msc-en}
    } & {
      \schlName{\theschlname(elach)}{\uppercase{elach}-UMINHO}{\theschlurl(elach)}
      \degreeName{All PhD Programs}{PhD}{https://www.uminho.pt/EN/education/educational-offer/Pages/PhD-degrees.aspx}
      \degreeName{All MSc Programs}{MSc}{https://www.uminho.pt/EN/education/educational-offer/Pages/Master-degrees.aspx}
    }\\
    {
      \docCover{uminho-i3b-phd-en,uminho-i3b-msc-en}
    } & {
      \schlName{\theschlname(i3b)}{\uppercase{i3b}-UMINHO}{\theschlurl(i3b)}
      \degreeName{All PhD Programs}{PhD}{https://www.uminho.pt/EN/education/educational-offer/Pages/PhD-degrees.aspx}
      \degreeName{All MSc Programs}{MSc}{https://www.uminho.pt/EN/education/educational-offer/Pages/Master-degrees.aspx}
    }\\
\end{ntUniversity}


%
% % \begin{ntUniversity}{ISCTE — Instituto Universitário de Lisboa}
% %     \ntSchool{cover-iscteiul-eta-phd}%
% %              {Escola{ de Tecnologias e Arquitectura};{ETA-ISCTE-IUL};{https://ciencia.iscte-iul.pt/schools/escola-tecnologias-arquitectura}%}
% % \end{ntUniversity}
%
\begin{ntUniversity}{Instituto Politécnico de Lisboa}
    {
      \docCover{ipl-isel-msc-en}
    } & {
      \schlName{Instituto Superior de Engenharia de Lisboa}{ISEL-IPL}{https://www.isel.pt}
      \degreeName{All MSc Programs}{MSc}{https://www.isel.pt/candidatos/candidaturas/mestrados}
    }\\
\end{ntUniversity}

\begin{ntUniversity}{Instituto Politécnico de Setúbal}
    {
      \docCover{ips-ests-msc-en}
    } & {
      \schlName{Escola Superior de Tecnologia de Setúbal}{ISEL-IPL}{https://www.estbarreiro.ips.pt}
      \degreeName{All MSc Programs}{MSc}{https://www.estbarreiro.ips.pt/candidaturas/mestrados}
    }\\
\end{ntUniversity}

\begin{ntUniversity}{Other Universities/Schools/Degrees}
    {
      \docCover{other-esep-msc-en}
    } & {
      \schlName{Escola Superior de Enfermagem do Porto}{ESEP}{https://www.esenf.pt/pt}
      \degreeName{All MSc Programs}{MSc}{https://estudar.esenf.pt/mestrados}
    }\\
\end{ntUniversity}


\section{Getting Started}
\label{sec:getting_started}

The template provides an \emph{easy to use} setting for you to write your thesis/dissertation in \LaTeX:
\begin{itemize}
  \item  Select your school;
  \item Fill your thesis metadata (title, research field, etc) in the file “\texttt{template.tex}”;
  \item Create your thesis/dissertation contents using the files in folder “\texttt{Chapters}”; and
  \item Process using you favorite \LaTeX\ processor (pdf\LaTeX, \XeLaTeX\ or \LuaLaTeX).
\end{itemize}

\subsection{Using Overleaf}
\label{sub:using_overleaf}

\ntindex[Installation!Overleaf]{}
\ntindex[Using!Overleaf]{}

\newcommand{\Overleaf}{\href{https://www.overleaf.com?r=f5160636&rm=d&rs=b}{Overleaf}}

\begin{wrapfigure}{r}{0.3\linewidth}
% \vspace*{-10ex}
\includegraphics[width=\linewidth]{overleaf}%
\caption{NOVAthesis template in Overleaf.}
\label{fig:overleaf}
\end{wrapfigure}
\mbox{}\Overleaf\ is a collaborative cloud-based LaTeX editor used for writing, editing and publishing scientific documents. Like “Google Docs”,  for \LaTeX\ users. You can edit and compile your \LaTeX\ source on the cloud, without installing software in your own computer, and, much like \emph{Google Docs}, you can share your document with others users and everybody can edit the same file at the same time (this may be dangerous).

If you do not have an account in \Overleaf, you must \href{https://www.overleaf.com?r=f5160636&rm=d&rs=b}{create one first}.

Once you have an account, please access the \gls{novathesis} template in \href{https://www.overleaf.com/latex/templates/novathesis-v7-dot-1-18/jhqwhtcwbmqc}{Overleaf} and select the green button \emph{Open as Template} (see \Autoref{fig:overleaf}).

\bgroup
  \itshape
  Please notice that the version currently available in Overleaf (v7.1.18) is slightly outdated (current version is v\novathesisversion). A new version (v7.1.29) will be submitted to Overleaf soon.  Until then, please:
  \begin{enumerate}
    \item Download the \href{https://github.com/joaomlourenco/novathesis/archive/main.zip}{latest version} from the GitHub repository as a Zip file.
    \item Login to your favorite LaTeX cloud service. I recommend \href{https://www.overleaf.com/?r=f5160636&rm=d&rs=b}{Overleaf} but there are alternatives (these instructions apply to Overleaf and you'll have to adapt for other providers).
    \item In the menu select: \texttt{New project} $\rightarrow$ \texttt{Upload project}.
    \item Upload the zip file.
    \item Select “template.tex” as the main file.
    \item Let Overleaf compile the document.
  \end{enumerate}
\egroup

\begin{tcolorbox}[colback=red!8]
	Notice that you need a (student) subscription to compile the \novathesis\ template in Overleaf, otherwise your compilation will always time out.
\end{tcolorbox}

\subsection{Using a Local \LaTeX\ Installation Local}
\label{sub:using_local_latex}

\ntindex[Installation!Local installation]{}
\ntindex[Using!Local installation]{}

\begin{wrapfigure}{r}{0.3\linewidth}
\vspace*{-7ex}
\includegraphics[width=\linewidth]{github}%
\caption{The NOVAthesis Project page in GitHub.}
\label{fig:github2}
\end{wrapfigure}

First of all, start by installing \LaTeX\ in your computer.  There are two main distributions, \href{https://miktex.org}{\ntindex{\MikTeX}}\ and \href{https://www.tug.org/texlive/}{\ntindex{\TeXLive}}, and both of them are available for the~3 most popular Operating Systems: Linux, macOS and Windows.

Be aware that a full installation of \MikTeX\ or \TeXLive\ will take near~5\,GB of hard disk space.  So, think twice before installing the full distribution.  See the \gls{novathesis} Wiki for the \href{https://github.com/joaomlourenco/novathesis/wiki/installing-latex#minimal-installation-in-any-of-the-systems-above}{list of packages required to compile the template}.

Once you have \LaTeX\ up and running, remember to install a good \LaTeX\ text editor.  I recommend you to take a look at  \href{https://tex.stackexchange.com/questions/339/latex-editors-ides}{this post} in the \url{tex.stackexchange.com} site.  If you want a quick and dirty recommendation, try \href{https://www.texstudio.org/}{\ntindex{TeXStudio}}.

Now, you must access the \gls{novathesis} repository in \href{https://github.com/joaomlourenco/novathesis}{GitHub}, select the green button \emph{Code} and then \emph{download} (or \emph{clone}) the template.  You will always get the latest version of the template (currently v\novathesisversion\ from \novathesisdate).


\section{Getting Help}
\label{sec:getting_help}

\ntindex[Help]{}

No! You don't have to use this template to write your thesis.  You don't even have to use \LaTeX.  However, writing a thesis is serious stuff, and which tool you shall use to write it is not a decision to make lighthearted.

\LaTeX\ is hard enough by itself.  This template aims at making your life easier, but not easy. If you choose to use this template to write your thesis, you are very welcome.  However, don't expect me to provide you help with \LaTeX.  Look for help with your friends (you have some friends, don't you?), or search the web, or try even to read some book(s) on \LaTeX. In the end you will certainly find the experience rewarding.

When you come to the point of “\emph{How do I do this with the \novathesis\ template?}”, remember…

\begin{enumerate}
  \item To check the \href{https://github.com/joaomlourenco/novathesis/wiki}{\gls{novathesis} wiki} and have some hope!  \emojiSmile
  \item \href{https://www.google.com}{Google} is your best friend.
  \item Search the \href{https://github.com/joaomlourenco/novathesis/discussions}{GitHub Discussions page} for a question related to yours.  \emph{If and only if} you don't find one, then post your own question in English please!
  \item Search the \href{https://www.facebook.com/groups/novathesis}{NOVAtheis Facebook group} for a question related to yours.  \emph{If and only if} you don't find one, then post your own question in either Portuguese or English, at your preference.
\end{enumerate}

When you post your own question, remember to \textbf{always} state the \gls{novathesis} version number you are using and referring to.

\begin{tcolorbox}[colback=blue!8]
	\centering
Please do not attempt to contact me directly (email, Messenger, etc)…\\I WILL NOT REPLY!
\end{tcolorbox}


\subsection{Suggestions, Bugs and Feature Requests} % (fold)
\label{sub:suggestions_bugs_and_feature_requests}

\begin{description}
  \item[Help:] If you just need some help, see above \Autoref{sec:getting_help}.
  \item[Suggestion:] \ntindex[Suggestions]{} Do you have a suggestion/recommendation? Please add it to the wiki and help other users!
  \item[Bug:] \ntindex[Bugs]{} Did you find a bug? Please open an issue. Thanks!
  \item[New Feature:] \ntindex[Feature Requests]{} Would you like to request a new feature (or support of a new School)? Please open an issue. Thanks!

\end{description}



% subsection suggestions_bugs_and_feature_requests (end)




\section{Donors}
\label{sec:donations}

\ntindex[Donations]{}

The \href{https://github.com/joaomlourenco/novathesis/wiki#donators}{list of \emph{Donnors}} is available in the \gls{novathesis} Project page.


\section{Disclaimer}
\label{sec:disclaimer}

\ntindex[Disclaimer]{}

Although the \gls{novathesis} template is endorsed by some Schools (e.g., \href{https://www.fct.unl.pt/estudante/informacao-academica/teses-e-dissertacoes}{linked from FCT-NOVA web site}), the \gls{novathesis} template \textbf{this not an official template} for any School.

The \gls{novathesis} template exists to make your life easier and we do our best to make it compliant to the supported ($+25$) Schools' regulations but, in the end of the line, you and only you are accountable for both the look and the contents of the document you submit as your thesis/dissertation.
