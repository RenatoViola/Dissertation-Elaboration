%!TEX root = ../template.tex
%%%%%%%%%%%%%%%%%%%%%%%%%%%%%%%%%%%%%%%%%%%%%%%%%%%%%%%%%%%%%%%%%%%
%% chapter1.tex
%% NOVA thesis document file
%%
%% Chapter with introduction
%%%%%%%%%%%%%%%%%%%%%%%%%%%%%%%%%%%%%%%%%%%%%%%%%%%%%%%%%%%%%%%%%%%

\typeout{NT FILE chapter1.tex}%

\chapter{Introduction}
\label{cha:introduction}

% epigraph configuration
\epigraphfontsize{\small\itshape}
\setlength\epigraphwidth{12.5cm}
\setlength\epigraphrule{0pt}

\epigraph{
  This chapter introduces the problems addressed by this dissertation and establishes the motivation for exploring spatial audio as a tool to improve accessibility in art museums. It outlines the objectives of the research, the expected contributions, and provides an overview of the proposed solution.
}

In a colorful world brimming with life and filled with the most varied shapes and patterns, each of our five senses is vital in obtaining information about our surroundings~\cite{five-senses}. Out of these, vision is by far the most dominant, as about 80\% of what we learn about the world and the very impressions we perceive are by means of our sight~\cite{vision-is-dominant-sense}. It is so valued in fact, that not only 77\% of people state that it is their most important sense, but would also rather live shorter lives than a longer one without their sight~\cite{sight-is-most-important-sense-1, sight-is-most-important-sense-2}.

Unfortunately, sight is not a universal privilege. According to the International Agency for the Prevention of Blindness (IAPB) Vision Atlas~\cite{global-estimates-of-vision-loss}, as of 2020, approximately 1.1 billion people worldwide were living with some form of vision loss. While over the last few decades there has been a decrease of vision loss prevalence in proportion across the population, the absolute numbers have increased over time and don't seem to be dwindling anytime soon. On the contrary, all categories of vision loss are set to grow as it is projected that by 2050, the world will have roughly 1.8 billion people experiencing visual impairment~\cite{vision-loss-projections-2020-2050}. This is an increase in vision loss of 55\%, or 600 million people over the next 30 years.

If inclusivity and disability rights alone aren’t enough motivation to care for this community, the World’s Health Organization (WHO)~\cite{who-vision-impairment} states that everyone, if they live long enough, will at least experience some sort of eye condition in their lifetime in need of proper care. As such, the concerns of the visually impaired community should in fact be everyone's concerns as well.


\section{Motivation}
\label{sec:motivation}

Art is a universal form of human expression, be it cultural, creative, somewhere in between, or something else entirely. From their very conception, musems have traditionally been one of, if not the most prominent way to not only access such art, but to celebrate and share it across generations. These institutions are mostly known for being the custodians of history, heritage and artistic expression.

It is the very mission of a museum to be an inclusive and accessible space of cultural participation, a shared human experience. However, that is not always the case. Exhibitions have historically been visual in nature, as most of the art made available is specifically designed to be consumed that way and due to numerous efforts in preserving it. This is a huge barrier for BVI people in terms of access to information and even hinders their mobility and independence within an exhibition~\cite{cavazos2021accessible, li2023understanding, vasilakou2022accessibility, vaz2020blind, vaz2020perspectives}.

Though the visually impaired are still not accommodated properly to this day, there have been strides in the right direction, one of which is in the form of legislation advocating for their right to participate in cultural life~\cite{holloway2019making,li2023understanding,martins2020blindness} – article 30 of the UN Convention on the Rights of Persons with Disabilities~\cite{crpd-rights}. 

In recent times, museums too have made considerable efforts in adapting to the needs of the visually impaired aside from mobility and navigation, gradually becoming more inclusive and participatory spaces~\cite{asakawa2019independent,holloway2019making,li2023understanding,martins2020blindness,rector2017eyes}. These efforts tend to be expressed as: tactile replicas and graphics, pre-recorded audio descriptions, specialized tours, workshops, large print and labels in braille, among several others. Though each of these methods serves a relevant and unique purpose, they are not without limitations, which will be enumerated in section~\ref{sec:objectives}.

Not to undermine the progress achieved over the last few years, as institutions are now more accessible than they ever were, most museums are still largely inaccessible to a BVI audience, as accessibility is more of an afterthought than a priority~\cite{candlin2003blindness,holloway2019making,rector2017eyes,vasilakou2022accessibility}. As a result, low-vision people very rarely attend these institutions - in Europe, only 5.5\% of them actually do~\cite{vaz2020blind,vaz2020perspectives}. This is particularly due to the aforementioned ocular centricity which greatly limits not only their mobility but also their access to information, which in turn greatly hinders their independence as well~\cite{cavazos2021accessible,li2023understanding,vasilakou2022accessibility}.

Such a low percentual for museum adherence is particularly demoralizing, as it is a common observation that BVI people actually do enjoy and express the desire to visit galleries and experience visual art~\cite{asakawa2019independent,candlin2003blindness,holloway2019making,krol2024design,li2023understanding}, and much for the same reasons people with sight do~\cite{candlin2003blindness}. However, they must be provided with the proper access to do so~\cite{holloway2019making} and don’t want to constantly rely on others~\cite{asakawa2019independent,krol2024design}.

While the numerous accessibility methods currently employed at museums merit their very welcome addition to any exhibition, these often tend to understate the importance of the aestheticism in experiencing the artwork~\cite{krol2024design,li2023understanding,martins2020blindness}. Instead, the focus leans heavily on describing and educating, sometimes at the cost of the sensory, emotional and immersive dimensions of a piece. 

Good accessibility requires a fine-tuned balance of both exposure to information and actual engagement with the artwork. This is especially important for visually impaired visitors, as it offers them the opportunity to develop meaningful connections to art, going beyond the pure intellectual understanding of it~\cite{martins2020blindness}. Approaches combining music and soundscapes have shown promise in this regard, but can be costly in terms of time and resources~\cite{krol2024design}.

Technological innovations such as spatial audio, haptics, VR and AR have enabled the creation of increasingly immersive experiences~\cite{chang2024sound,li2024beyond,sanchez2007usability,yang2019audio}, allowing not only for multisensory engagement with art but also opportunities for remote artistic appreciation. Despite the risk of less engagement than in person~\cite{li2023understanding} and difficulty navigating through a web gallery~\cite{virtual-space-accessibility}, digital platforms and virtual tours can tear down physical barriers by enabling individuals to experience art from the comfort of home or anywhere else, at their own pace, and unshackled by social tension~\cite{how-museums-remove-barriers-for-bvi,li2023understanding}. 

The potential of virtual environments to provide a variety of interaction modalities and content has been thoroughly investigated, and sound-based approaches are especially important for blind and visually impaired people~\cite{sanchez2007usability,yang2019audio}. By mimicking environmental cues, spatial audio improves spatial orientation and makes it easier for users to move freely and independently in virtual environments, helping them visualize virtual worlds~\cite{sanchez2007usability}. Frequently enhanced with 3D effects to improve spatial perception and experience, 3D audio is particularly useful for navigation and immersive engagement in museum and gallery settings~\cite{yang2019audio}.

Mostly supporting the above-mentioned technologies, the smartphone is the perfect vehicle for accessibility as about 54\% of the global population owns at least one~\cite{smartphone-stats}, and it is rich in accessibility features among several others~\cite{smartphone-features}. This study draws inspiration from several implementations of accessible technologies to develop a remote mobile and 3D sound-based solution, that addresses both the informational and aesthetic needs of BVI visitors.



\section{Problem Description \& Objectives}
\label{sec:objectives}

As it was briefly alluded to in section~\ref{sec:motivation}, there are some limitations to the accessibility methods usually active at museums~\cite{candlin2003blindness,cavazos2021accessible,holloway2019making,rector2017eyes}. Without delving into too much detail, tours and workshops are infrequent and inconsistent, must be reserved in advance and are only available on specific dates or time slots. Though more common in museums, audio descriptions are primarily designed with normovisual people in mind, focusing mostly on interpretation and historical context, not accessibility. Braille-based brochures leave a great deal of relevant information aside, and braille proficiency is generally low~\cite{cavazos2021accessible}.

While a tactile approach seems to be the preferred form of interaction with artwork for BVI individuals~\cite{krol2024design,li2023understanding}, since it allows them to feel the artwork “up close and personal” sensing its various features at a low level, high level information about the piece is quite limited and the combination of preservation efforts and barriers of intellectual access with the still prevalent visual centricity of exhibitions makes it so that these types of programs are a rare occurrence among museums~\cite{li2023understanding}.

In the context of this dissertation, we intend to tap into the potential of spatial audio as a means of creating an immersive remote experience that can convey the spatial and emotional dimensions of art, through interactive three-dimensional soundscapes. To achieve this, we draw inspiration from the use of spatial audio in other research as well as in video games and their accessibility features. Although video games may not be the first medium that comes to mind when considering the experience of visual art in museums, they share certain parallels. As an art form and medium of expression of its own, video games focus on immersion and engagement, much like museums aim to captivate their visitors. Over time, video games have also evolved to become more accessible and appeal to a broader audience through the innovative use of spatial audio to guide navigation, evoke emotion and tell a rich, interactive story. This aligns closely with our objectives and while our proposal is not to develop a video game, it adopts a gamified approach to reimagining how art can be experienced inclusively.

Thus, we propose the creation of a tool allowing museum curators to build a spatial audio environment representing art pieces, enabling the staff to define which parts of the environment are explorable by the visitors. In essence, it is an intuitive and user-friendly soundscape editor with support for immersive audio and interactivity, not requiring prior experience in more sophisticated tools with a harsh learning curve. The generated environments are then made available via an APK for Android devices, where blind and visually impaired clients can navigate a simplified top-down map-like view of the scene with contrasting elements, using virtual thumb joysticks to define direction and movement. One such joystick is implemented as a directional scanner, for precise exploration. With only a smartphone coupled with a pair of headphones, any user is able to explore immersive 3D sound simulating the experience of physically approaching or moving away from different parts of the artwork, using controls that are both basic and familiar.


\section{Expected Contributions}

This research is expected to yield the following contributions:
\begin{itemize}
  \item An intuitive soundscape editor for museum curators to create spatial audio environments representing art pieces, requiring no prior expertise in complex audio design.
  \item An APK (Android Package Kit) enabling BVI users to explore artworks through immersive 3D audio, providing an experience that is both aesthetic and informational while promoting independence with simple controls.
  \item A cost-effective approach to accessibility using widely available technologies, such as smartphones and headphones, allowing for both remote and on-site interaction with art.
  \item Validation through usability testing and feedback from BVI users, in order to evaluate the solution's effectiveness in real-world scenarios.
\end{itemize}

\section{Document Structure}

There are four main chapters to this document:
\begin{itemize}
  \item \textbf{Chapter 1 - Introduction:} Introduces the problems addressed by this dissertation and establishes the motivation for exploring spatial audio as a tool to improve accessibility in art museums. It outlines the objectives of the research, the expected contributions, and provides an overview of the proposed solution.
  \item \textbf{Chapter 2 - Background:} Presents the foundational concepts most relevant in understanding the work to be developed. It covers the principles of sound from its definition, perception, transmission and spatial localization. Additionally, the definition and purpose of soundscapes are addressed and most importantly the role these play in accessibility for the blind and visually impaired. These concepts provide the theoretical grounding for the proposed solution.
  \item \textbf{Chapter 3 - Related Work:} Reviews relevant research and applications mostly related to immersive spatial audio and BVI accessibility, including technologies and approaches that have been used to address said accessibility. It is divided by sections, each of which focusing on a specific topic related to the dissertation's theme. Each of these sections starts with a brief overview of related studies and projects displaying the current state of the art. It is then followed by subsections where projects that are of particular relevance to this dissertation are explored in some detail, from implementation to findings, and finally how they relate to our work and what is to be learned from them.
  \item \textbf{Chapter 4 - Proposed Solution:} Describes the proposed system's architecture in some detail, addressing the expected technological stack, the envisioned plan for its development and the methodology for validating the solution. The work plan/schedule is mapped in a Gantt chart.
\end{itemize}

\newcommand{\Overleaf}{Overleaf}