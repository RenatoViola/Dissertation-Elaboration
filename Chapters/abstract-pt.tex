%!TEX root = ../template.tex
%%%%%%%%%%%%%%%%%%%%%%%%%%%%%%%%%%%%%%%%%%%%%%%%%%%%%%%%%%%%%%%%%%%%
%% abstract-pt.tex
%% NOVA thesis document file
%%
%% Abstract in Portuguese
%%%%%%%%%%%%%%%%%%%%%%%%%%%%%%%%%%%%%%%%%%%%%%%%%%%%%%%%%%%%%%%%%%%%

\typeout{NT FILE abstract-pt.tex}%

Na sua missão como espaços inclusivos de participação e celebração cultural, os museus têm dado passadas consideráveis para ultrapassar a sua dependência histórica da visão e acomodar adequadamente as necessidades dos cegos e deficientes visuais. 

No entanto, a acessibilidade tende a ser uma reflexão tardia e não uma prioridade, e a maioria das exposições tem permanecido inacessível a um público invisual e amblíope. Por muito meritórios que sejam, os métodos de acessibilidade mais comuns não conseguem muitas vezes equilibrar a exposição à informação e o envolvimento adequado com a obra de arte, se disponíveis sequer. Por verem a sua independência, mobilidade e acesso interpretativo condicionados, as pessoas cegas e com baixa visão raramente frequentam estas instituições, apesar de apreciarem e manifestarem o desejo de experienciar a arte visual.

Nesta dissertação, propomos uma abordagem remota e interativa para a representação de arte visual acessível a pessoas com deficiência visual, em que o áudio espacial proporciona uma sensação de imersão e simula a exploração. O sistema concebido divide-se simbióticamente, em que cada parte se destina a um público único.

O sistema inclui um editor de paisagens sonoras em 3D para \textit{Windows}, onde curadores de museus sem conhecimentos prévios de \textit{design} de som podem rapidamente desenvolver cenas auditivas virtuais e imersivas, representativas de obras de arte específicas. Os ambientes gerados são interativos mediante um leitor de paisagens sonoras móvel, onde os utilizadores com deficiência visual exploram autonomamente a composição das cenas através de movimento assistido por um \textit{thumbstick}, sinais direcionais de áudio e uma ferramenta de análise da vizinhança. A experiência oferece espaço para alguma personalização, incorporando controlos para regulação da carga sensorial.

Fundamentalmente, a nossa proposta visa satisfazer as necessidades informativas e estéticas dos visitantes de baixa visão no que respeita ao acesso às artes visuais, promovendo a sua independência e mantendo-se eficaz em custo.

% E agora vamos fazer um teste com uma quebra de linha no hífen a ver se a \LaTeX\ duplica o hífen na linha seguinte se usarmos \verb+"-+… em vez de \verb+-+.
%
% zzzz zzz zzzz zzz zzzz zzz zzzz zzz zzzz zzz zzzz zzz zzzz zzz zzzz zzz zzzz comentar"-lhe zzz zzzz zzz zzzz
%
% Sim!  Funciona! :)

% Palavras-chave do resumo em Português
% \begin{keywords}
% Palavra-chave 1, Palavra-chave 2, Palavra-chave 3, Palavra-chave 4
% \end{keywords}
\keywords{
  Áudio Espacial \and
  Cegos e Amblíopes \and
  Cultura Acessível \and
  HRTF \and
  Paisagem Sonora
}
% to add an extra black line
