%!TEX root = ../template.tex
%%%%%%%%%%%%%%%%%%%%%%%%%%%%%%%%%%%%%%%%%%%%%%%%%%%%%%%%%%%%%%%%%%%%
%% abstract-pt.tex
%% NOVA thesis document file
%%
%% Abstract in Portuguese
%%%%%%%%%%%%%%%%%%%%%%%%%%%%%%%%%%%%%%%%%%%%%%%%%%%%%%%%%%%%%%%%%%%%

\typeout{NT FILE abstract-pt.tex}%

Na sua missão como espaços inclusivos de participação e celebração cultural, os museus têm dado passadas consideráveis para ultrapassar a sua dependência histórica da visão e acomodar adequadamente as necessidades das pessoas cegas e com baixa visão. 

No entanto, a acessibilidade tende a ser uma reflexão tardia e não uma prioridade, e a maioria das exposições tem permanecido inacessível a um público com deficiência visual. Por meritórios que sejam, os métodos de acessibilidade mais comuns não conseguem muitas vezes equilibrar a exposição à informação e o devido envolvimento com a obra de arte, se disponíveis sequer. Por verem a sua independência, mobilidade e acesso interpretativo condicionados, na Europa, as pessoas cegas e com baixa visão raramente frequentam estas instituições, apesar de apreciarem e manifestarem o desejo de experienciar a arte visual.

Nesta dissertação, é proposta uma abordagem híbrida quanto à interpretação acessível de arte visual para atenuar barreiras de acesso atuais, proporcionando interação com a recriação virtual sonora de uma pintura através do acesso sensorial e descritivo ao seu conteúdo. Está dividida simbióticamente em dois componentes direcionados a públicos distintos: pessoas com deficiência visual e curadores ou profissionais de arte.

O sistema concebido inclui um reprodutor de paisagens sonoras móvel, onde os utilizadores invisuais exploram autonomamente ambientes auditivos representativos de pinturas, através de movimento assistido por um \textit{thumbstick}, sinais direcionais de áudio e uma ferramenta de análise da vizinhança, entre outras ferramentas de noção espacial. A aplicação é complementada por um editor de paisagens sonoras para \textit{desktop}, no qual curadores sem experiência prévia em \textit{design} sonoro, podem hábilmente desenvolver as cenas auditivas imersivas disponibilizadas no reprodutor de paisagens sonoras para

Fundamentalmente, esta proposta visa satisfazer as necessidades informativas e estéticas dos visitantes cegos ou de baixa visão no que respeita ao acesso à arte visual, promovendo a sua independência e derrubando barreiras à acessibilidade. O sistema foi avaliado através de um estudo de utilizador distinto para cada um dos seus componentes, no qual os participantes avaliaram as funcionalidades e a viabilidade geral da aplicação.

% E agora vamos fazer um teste com uma quebra de linha no hífen a ver se a \LaTeX\ duplica o hífen na linha seguinte se usarmos \verb+"-+… em vez de \verb+-+.
%
% zzzz zzz zzzz zzz zzzz zzz zzzz zzz zzzz zzz zzzz zzz zzzz zzz zzzz zzz zzzz comentar"-lhe zzz zzzz zzz zzzz
%
% Sim!  Funciona! :)

% Palavras-chave do resumo em Português
% \begin{keywords}
% Palavra-chave 1, Palavra-chave 2, Palavra-chave 3, Palavra-chave 4
% \end{keywords}
\keywords{
  Cegos e Baixa Visão \and
  Cultura Acessível \and
  Áudio Espacial \and
  Paisagem Sonora
}
% to add an extra black line
