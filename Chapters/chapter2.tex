%!TEX root = ../template.tex
%%%%%%%%%%%%%%%%%%%%%%%%%%%%%%%%%%%%%%%%%%%%%%%%%%%%%%%%%%%%%%%%%%%%
%% chapter2.tex
%% NOVA thesis document file
%%
%% Chapter with the template manual
%%%%%%%%%%%%%%%%%%%%%%%%%%%%%%%%%%%%%%%%%%%%%%%%%%%%%%%%%%%%%%%%%%%%

\typeout{NT FILE chapter2.tex}%

\chapter{Background}
\label{cha:background}

% epigraph configuration
\epigraphfontsize{\small\itshape}
\setlength\epigraphwidth{12.5cm}
\setlength\epigraphrule{0pt}

\epigraph{
  This chapter presents the foundational concepts necessary to understand the work. It covers the principles of sound transmission and localization, binaural hearing, immersive audio, human-computer interaction and soundscapes. These concepts provide the theoretical grounding for the proposed solution.
}{}

\glsresetall

\section{Sound Perception}
\label{sec:sound-perception}

\subsection{What Is Sound And How Is It Perceived?}
\label{ssec:what-is-sound}

% talk about BVI people perceive sounds differently
\subsection{BVI People Experience Sound Differently}
\label{ssec:bvi-experience-sound}


\section{Binaural Hearing}
\label{sec:binaural-hearing}

\subsection{Binaural Audio}
\label{ssec:binaural-audio}

\subsection{Sound Localization}
\label{ssec:sound-localization}

\subsubsection{Interaural Differences}
\label{sssec:interaural-differences}

\subsubsection{The Confusion Cone}
\label{sssec:confusion-cone}

\subsubsection{Head Related Transfer Function (HRTF)}
\label{sssec:hrtf}

\section{Immmersive Audio}
\label{sec:immersive-audio}

\subsection{Spatial Audio}
\label{ssec:spatial-audio}

\subsection{Stereo VS Surround Sound}
\label{ssec:stereo-vs-surround}

\subsection{Techniques For Rendering 3D Audio}
\label{ssec:rendering-3d-audio}

\subsubsection{Ambisonics}
\label{sssec:ambisonics}

\subsubsection{Wave Field Synthesis (WFS)}
\label{sssec:wfs}

\section{Human-Computer Interaction}
\label{sec:hci}

\section{Soundscape}
\label{sec:soundscape}